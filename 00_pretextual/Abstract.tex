\begin{center}
\textbf{ABSTRACT}
\end{center}

$\!$\\

%Este trabalho busca aplicar o algoritmo de programação genética em problemas formulados com base nos conceitos da aprendizagem por reforço. São utilizadas duas bibliotecas na linguagem de programação Python: DEAP e Gym, que implementam algoritmos evolucionários distribuídos e ambientes de simulação inspirados em processos de decisão de Markov, respectivamente. Foi utilizado como exemplo, ao longo do trabalho, a aplicação da proposta no problema clássico de controle do pêndulo invertido. O método foi aplicado em outros problemas propostos, incluindo uma aplicação real que utiliza um carro robô, construído por alunos do curso de engenharia elétrica da UERJ.

This work seeks to apply the genetic programming algorithm to problems formulated with reinforcement learning concepts. Two libraries are used in the Python programming language: DEAP and Gym, which implements distributed evolutionary algorithms and simulation environments inspired by Markov's decision processes, respectively. As an example, throughout the work, the classic case of the control of an inverted pendulum was used. The method was also applied to other systems, including a real application that uses an autonomous vehicle, built by students of the electrical engineering course at the Rio de Janeiro State University.

\vspace{1cm}

\hspace{-1.3cm}Keywords: Genetic Programming, Reinforcement Learning.