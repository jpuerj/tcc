\begin{center}
\textbf{RESUMO}
\end{center}

%
% O resumo deve ser organizado em apenas um parágrafo mesmo.
% O número de folha é o número de páginas do PDF -2. Isto ocorre pois na versão final (capa dura) a capa é removida e as duas primeiras páginas são impressas em uma % folha apenas (frente e verso).
%

$\!$\\

\hspace{-1.3cm}\textbf{Ferreira}, J.P.B. \textit{Programação Genética Aplicada a Problemas de Aprendizagem por Reforço}. 127 f. Projeto Final de Curso (Graduação em Engenharia Elétrica) - Faculdade de Engenharia, Universidade do Estado do Rio de Janeiro, Rio de Janeiro, 2020.

\vspace{.2cm}

Este trabalho busca aplicar o algoritmo de programação genética em problemas formulados com base nos conceitos da aprendizagem por reforço. São utilizadas duas bibliotecas na linguagem de programação Python: DEAP e Gym, que implementam algoritmos evolucionários distribuídos e ambientes de simulação inspirados em processos de decisão de Markov, respectivamente. Foi utilizado como exemplo, ao longo do trabalho, a aplicação da proposta no problema clássico de controle do pêndulo invertido. O método foi aplicado em outros sistemas, incluindo uma aplicação real que utiliza um veículo terrestre autônomo, construído por alunos do curso de engenharia elétrica da UERJ.

\vspace{1cm}

\hspace{-1.3cm}Palavras-chave: Programação Genética, Aprendizagem por Reforço.