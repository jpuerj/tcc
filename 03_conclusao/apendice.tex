\appendix
\section*{APÊNDICES}
\addcontentsline{toc}{section}{\bfseries{APÊNDICES}}
\addtocontents{toc}{\begingroup\string\makeatletter\global\let\string\l@section\string\l@subsection\endgroup}

\section{Códigos}\label{apendice:codigos}

Todos os códigos utilizados neste trabalho podem ser encontrados no repositório abaixo:

\url{https://github.com/jpuerj/gp-openai-gym}

\section{Condições Inicias}\label{apendice:cond-iniciais}

As condições iniciais de um ambiente são determinadas pela faixa de valores para as quais as variáveis de estado podem ser inicializadas, indicando o início de um episódio. Todos os valores aleatórios são gerados a partir de uma distribuição probabilística uniforme.

\subsection{Pêndulo Invertido}

\begin{table}[H]
	\centering
	\caption{Condições iniciais do ambiente \textit{CartPole-v1}.}
	\label{tab:ap-cartpoleci}
	\begin{tabular}{SSS} \toprule
		{Variável} & {Valor Mínimo} & {Valor Máximo} \\ \midrule
		{$s$} & {-0,05} & {0,05} \\
		{$\dot{s}$} & {-0,05} & {0,05} \\
		{$v$} & {-0,05} & {0,05} \\
		{$\dot{v}$} & {-0,05} & {0,05} \\
		\bottomrule
	\end{tabular}
\end{table}

\subsection{Pêndulo Swing-up}

\begin{table}[H]
	\centering
	\caption{Condições iniciais do ambiente \textit{Pendulum-v0}.}
	\label{tab:ap-pendulumci}
	\begin{tabular}{SSS} \toprule
		{Variável} & {Valor Mínimo} & {Valor Máximo} \\ \midrule
		{$\cos(\theta)$} & {-1} & {1} \\
		{$\sin(\theta)$} & {-1} & {1} \\
		{$\dot{\theta}$} & {-8} & {8} \\
		\bottomrule
	\end{tabular}
\end{table}

% \subsection{Pêndulo Duplo Invertido}

\subsection{Carro na Ladeira}

\begin{table}[H]
	\centering
	\caption{Condições iniciais do ambiente \textit{Pendulum-v0}.}
	\label{tab:ap-mcci}
	\begin{tabular}{SSS} \toprule
		{Variável} & {Valor Mínimo} & {Valor Máximo} \\ \midrule
		{$s$} & {-0,6} & {-0,4} \\
		{$\dot{s}$} & {0} & {0} \\
		\bottomrule
	\end{tabular}
\end{table}

\section{Hardware e Software Utilizados}\label{apendice:hardware}

\begin{table}[H]
	\centering
	\caption{Características do computador e dos programas utilizados para a execução do algoritmo.}
	\label{tab:ap-hardware}
	\begin{tabular}{SS} \toprule
		\multicolumn{2}{c}{Componente / Programa} \\ \midrule
		{Processador} & {Ryzen 5 1600}\\
		{Memória} & {Corsair Vengeance LPX 2x8GB DDR4 3000Mhz} \\
		{Placa-Mãe} & {Asrock A320M-HD} \\
		{Interpretador de Python} & {Anaconda} \\
		{IDE} & {Pycharm Community} \\
		\bottomrule
	\end{tabular}
\end{table}